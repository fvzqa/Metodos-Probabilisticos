\documentclass[final,6p,times,twocolumn]{elsarticle}
\makeatletter

\renewenvironment{abstract}{\global\setbox\absbox=\vbox\bgroup
	\hsize=\textwidth\def\baselinestretch{1}%
	\noindent\unskip\textbf{Resumen}  % <--- Edit as necessary
	\par\medskip\noindent\unskip\ignorespaces}
{\egroup}

\def\keyword{%
	\def\sep{\unskip, }%
	\def\MSC{\@ifnextchar[{\@MSC}{\@MSC[2000]}}
	\def\@MSC[##1]{\par\leavevmode\hbox {\it ##1~MSC:\space}}%
	\def\PACS{\par\leavevmode\hbox {\it PACS:\space}}%
	\def\JEL{\par\leavevmode\hbox {\it JEL:\space}}%
	\global\setbox\keybox=\vbox\bgroup\hsize=\textwidth
	\normalsize\normalfont\def\baselinestretch{1}
	\parskip\z@
	\noindent\textit{Palabras clave: }  % <--- Edit as necessary
	\raggedright                         % Keywords are not justified.
	\ignorespaces}


\def\ps@pprintTitle{%
	\let\@oddhead\@empty
	\let\@evenhead\@empty
	\def\@oddfoot{\footnotesize\itshape
		\ifx\@journal\@empty   % <--- Edit as necessary
		\else\@journal\fi\hfill\today}%
	\let\@evenfoot\@oddfoot}


\usepackage[spanish]{babel}
%\usepackage{hyperref}
%% The graphicx package provides the includegraphics command.
\usepackage{graphicx}
%% The amssymb package provides various useful mathematical symbols
\usepackage{amssymb}
%% The amsthm package provides extended theorem environments
%% \usepackage{amsthm}
\usepackage{natbib}
\usepackage{subcaption}

%% The lineno packages adds line numbers. Start line numbering with
%% \begin{linenumbers}, end it with \end{linenumbers}. Or switch it on
%% for the whole article with \linenumbers after \end{frontmatter}.
\usepackage{lineno}


%% natbib.sty is loaded by default. However, natbib options can be
%% provided with \biboptions{...} command. Following options are
%% valid:

%%   round  -  round parentheses are used (default)
%%   square -  square brackets are used   [option]
%%   curly  -  curly braces are used      {option}
%%   angle  -  angle brackets are used    <option>
%%   semicolon  -  multiple citations separated by semi-colon
%%   colon  - same as semicolon, an earlier confusion
%%   comma  -  separated by comma
%%   numbers-  selects numerical citations
%%   super  -  numerical citations as superscripts
%%   sort   -  sorts multiple citations according to order in ref. list
%%   sort&compress   -  like sort, but also compresses numerical citations
%%   compress - compresses without sorting
%%
%% \biboptions{comma,round}

% \biboptions{}

%\journal{Journal Name}

\begin{document}
	
	\begin{frontmatter}
		
		%% Title, authors and addresses
		
		\title{Sistema multiagente: simulación de una epidemia}
		
		%% use the tnoteref command within \title for footnotes;
		%% use the tnotetext command for the associated footnote;
		%% use the fnref command within \author or \address for footnotes;
		%% use the fntext command for the associated footnote;
		%% use the corref command within \author for corresponding author footnotes;
		%% use the cortext command for the associated footnote;
		%% use the ead command for the email address,
		%% and the form \ead[url] for the home page:
		%%
		%% \title{Title\tnoteref{label1}}
		%% \tnotetext[label1]{}
		%% \author{Name\corref{cor1}\fnref{label2}}
		%% \ead{email address}
		%% \ead[url]{home page}
		%% \fntext[label2]{}
		%% \cortext[cor1]{}
		%% \address{Address\fnref{label3}}
		%% \fntext[label3]{}
		
		
		%% use optional labels to link authors explicitly to addresses:
		%% \author[label1,label2]{<author name>}
		%% \address[label1]{<address>}
		%% \address[label2]{<address>}
		
		\author{Fabiola Vázquez}
		
		\address{Facultad de Ingeniería Mecánica y Eléctrica, UANL}
		
		\begin{abstract}
			%% Text of abstract
			Durante la historia de la humanidad, esta se ha enfrentado a diversas enfermedades que terminan con la vida de gran parte de la población. Debido a esto, durante ya muchos años se han desarrollado innumerables modelos matemáticos en busca de comprender estos fenómenos y mitigar sus daños. Un enfoque reciente es el uso de la simulación de sistemas multiagente, que permite ajustar parámetros a nivel individual en vez de poblacional, y medir los resultados de adoptar distintas acciones para mitigar contagios. El presente trabajo muestra un modelo de contagio multiagente, donde se estudia el efecto que pudiera tener la probabilidad de contagio sobre el porcentaje máximo de individuos infectados. Además se estudia el efecto de adoptar medidas como el uso de cubrebocas o vacunación.
		\end{abstract}
		
		\begin{keyword}
			Sistema multiagente \sep epidemia  \sep simulación \sep SIR 
			%% keywords here, in the form: keyword \sep keyword
			
			%% MSC codes here, in the form: \MSC code \sep code
			%% or \MSC[2008] code \sep code (2000 is the default)
			
		\end{keyword}
		
	\end{frontmatter}
	
	%%
	%% Start line numbering here if you want
	%%
	%\linenumbers
	
	%% main text
	\section{Introducción}
	Actualmente, a nivel global se vive con la pandemia ocasionada por el virus SARS-CoV-2 y algunas de las medidas para evitar su propagación son el uso correcto del cubrebocas, lavado de manos frecuentes y aislamiento social, es decir, no salir a menos que sea totalmente necesario. Muchas personas han hecho caso omiso a las indicaciones y continuamente salen sin protección alguna. Los objetivos de explorar este modelo son ver el impacto que tiene lo contagioso que es un virus, medido aquí como probabilidad de contagio, y mostrar el efecto que se puede obtener si se hace uso correcto de medidas preventivas como el cubrebocas. 
	
	En la sección \ref{S:2} se explican algunos conceptos básicos, como agente, epidemia, sistema multiagente y el modelo SIR. La sección \ref{S:trela} se detalla un poco sobre los trabajos que se han realizado anteriormente sobre los sistemas multiagentes, dentro y fuera del área de epidemiología. La sección \ref{S:Sprop} detalla el modelo planteado y los resultados obtenidos.
	
	\label{S:1}
	
	
	%\begin{itemize}
	%\item Bullet point one
	%\item Bullet point two
	%\end{itemize}
	%
	%\begin{enumerate}
	%\item Numbered list item one
	%\item Numbered list item two
	%\end{enumerate}
	
	
	
	
	
	%\begin{table}[h]
	%\centering
	%\begin{tabular}{l l l}
	%\hline
	%\textbf{Treatments} & \textbf{Response 1} & \textbf{Response 2}\\
	%\hline
	%Treatment 1 & 0.0003262 & 0.562 \\
	%Treatment 2 & 0.0015681 & 0.910 \\
	%Treatment 3 & 0.0009271 & 0.296 \\
	%\hline
	%\end{tabular}
	%\caption{Table caption}
	%\end{table}
	
	
	
	
	%\begin{figure}[h]
	%\centering\includegraphics[width=0.4\linewidth]{placeholder}
	%\caption{Figure caption}
	%\end{figure}
	
	
	
	
	
	\section{Antecedentes}
	Existen diversas maneras de definir un agente. Según la Real Academia Española, un agente es una persona o cosa que produce un efecto. Una de las características principales que posee un agente es su autonomía, es decir, que poseen la capacidad de tomar decisiones independientemente sin intervención alguna de uno o más agentes. Además poseen capacidad de inferencia, es decir que son capaces de observar de forma general la información. En los sistemas que se muestran, cada uno de los agentes \cite{introductionmas} posee sus propias características, como lo puede ser la posición inicial, su estado actual, si posee enfermedades previas, etcétera.   
	
	Un sistema multiagente \cite{surveymas} es un sistema donde un grupo de agentes autónomos interactúan en un entorno. Este tipo de sistemas han sido adaptados en diversas áreas debido a que, cuando se utiliza cómputo paralelo, este tipo de sistemas incrementa la velocidad y la eficiencia de la simulación.
	
	Un modelo epidemiológico de tipo SIR es un modelo con compartimentos donde cada agente puede estar en uno de tres estados: susceptible, infectado o recuperado. El agente solo puede pasar de susceptible a infectado (al ser contagiado), o de infectado a recuperado, donde se asume desarrolla inmunidad a la infección. La figura \ref{sir} muestra los cambios del estado del agente.
	\begin{figure}
		\label{sir}
		\centering
		\includegraphics[scale=0.2]{Images/sir.png}
		\caption{Modelo SIR.}
	\end{figure}
	
	
	
	
	
	\label{S:2}
	
	
	
	%% The Appendices part is started with the command \appendix;
	%% appendix sections are then done as normal sections
	%% \appendix
	
	%% \section{}
	%% \label{}
	
	%% References
	%%
	%% Following citation commands can be used in the body text:
	%% Usage of \cite is as follows:
	%%   \cite{key}          ==>>  [#]
	%%   \cite[chap. 2]{key} ==>>  [#, chap. 2]
	%%   \citet{key}         ==>>  Author [#]
	
	\section{Trabajos relacionados}
	\label{S:trela}
	Las investigaciones que involucran el uso de agentes comenzaron a principios de los años ochenta. En el año de 1990, Varian \cite{Varian} investigó un problema del ámbito económico donde los agentes pueden monitorear el comportamiento de otros agentes. 
	En teoría de juegos el uso de sistemas multiagentes es muy usado, por ejemplo, Pendharkar  \cite{gametheoretical} estudió el comportamiento cooperativo y competitivo de los agentes y trabajó también con conceptos del área de economía. Problemas de logística también han sido abordados con técnicas de simulación multiagente \cite{Wojtusiak_Warden_Herzog_2012, horl_2017}. En el área de epidemiología se tienen trabajos como los de \citet{Perez_Dragicevic_2009, Venkatramanan_Lewis_Chen_Higdon_Vullikanti_Marathe_2018, Hoertel_Blachier_Blanco_Olfson_Massetti_Rico_Limosin_Leleu_2020}. Swarup \cite{Swarup} describe  algunos problemas que aún no han sido resueltos en esta área. 
	\section{Solución propuesta}
	\label{S:Sprop}
	Se trabaja con el software libre R \cite{R} en un cuaderno de Jupyter \cite{jupyter}. El entorno de los agentes se considera como el cuadrado de lado 1.5 $\times$ 1.5 donde estos se posicionan uniformemente pseudo al azar. Se consideran diversos experimentos, en cada uno se trabajan con 50 agentes y un tiempo de 100. En la figura \ref{paso1} se muestra el estado inicial de los agentes en el entorno, donde los agentes de color rojo son los infectados, los verdes son los susceptibles y los triángulos invertidos son los agentes que están vacunados. 
	
	\begin{figure}
		\label{paso1}
		\centering
		\includegraphics[scale=0.4]{Images/p6_t001.png}
		\caption{Estado inicial de los agentes en su entorno.}
	\end{figure}
	
	En el primer experimento, los agentes pueden ser vacunados con una probabilidad \texttt{pv} y estos adquieren el estado de recuperado y no puede ser infectado nuevamente. En el segundo experimento, los agentes también pueden ser vacunados, pero se considera la probabilidad de 0.05 de que la vacuna falle, es decir, si el agente está vacunado y éste entra en contacto con alguien infectado, existe la posibilidad de volverse a contagiar; además se incluye que los agentes usan o no un cubrebocas para disminuir los contagios, esto se decide pseudo al azar. 
	\begin{figure}
		\label{sinmasc}
		\centering
		\includegraphics[scale=0.4]{Images/sinmasc.png}
		\caption{Ejemplo de una corrida considerando que los agentes se pueden vacunar, en este caso no se contempla el uso de cubrebocas.}
	\end{figure}
	\begin{figure}
		\label{conmasc}
		\centering
		\includegraphics[scale=0.4]{Images/conmasc.png}
		\caption{Ejemplo de una corrida considerando que los agentes vacunados pueden ser nuevamente susceptibles, en este caso se contempla el uso de cubrebocas.}
	\end{figure}
	La figura \ref{sinmasc} muestra el comportamiento de la epidemia en una sola corrida para el experimento 1 y la figura \ref{conmasc} muestra dicho comportamiento pero referente al experimento 2. Con una sola corrida no se puede ver el cambio que hay si el agente usa cubrebocas o no. 
	
	
	Se realiza la simulación variando la probabilidad \texttt{pv} de que un agente se pueda vacunar para ver el comportamiento en los dos experimentos mencionados anteriormente. La figura \ref{sinmasc1} muestra los gráficos de caja obtenidos con el experimento 1 y la figura \ref{conmasc1} los del experimento 2. Como se puede apreciar en dichas figuras, el comportamiento es similar, mientras mayor sea la probabilidad de que el agente se vacune, menor será el porcentaje de infectados. La figura \ref{conmasc1} comienza con un porcentaje mayor de infectados comparado con la figura \ref{sinmasc1} y esto es debido que en el segundo experimento, aún vacunado el agente, existe la probabilidad de que esta falle y sea contagiado. 
	
	\begin{figure}
		\label{sinmasc1}
		\centering
		\includegraphics[width=\linewidth]{Images/boxplot1.png}
		\caption{Gráficos de caja variando la probabilidad de que un agente se vacune en el experimento 1.}
	\end{figure}
	
	\begin{figure}
		\label{conmasc1}
		\centering
		\includegraphics[width=\linewidth]{Images/boxplot2.png}
		\caption{Gráficos de caja variando la probabilidad de que un agente se vacune en el experimento 2.}
	\end{figure}
	
	Se realiza un tercer estudio, considerando las condiciones del experimento 2 para determinar si hay un efecto si los agentes usan un cubrebocas o no. La probabilidad de que estos no lo posean se varía de 0 a 1 en saltos de 0.25. La figura \ref{conmasc2} muestra los gráficos de caja obtenidos en dicha simulación, donde se observa que si nadie usa cubrebocas los contagios aumentan.
	\begin{figure}
		\label{conmasc2}
		\centering
		\includegraphics[width=\linewidth]{Images/boxplot3.png}
		\caption{Gráficos de caja variando la probabilidad de que un agente no use cubrebocas en el experimento 2.}
	\end{figure}

	Por último, consideramos el efecto que puede tener lo contagioso que es un virus. Para esto, se varió la probabilidad $p$ de que un agente fuese contagiado por otro una vez que hicieran contacto. Las figuras \ref{fig:p25}, \ref{fig:p50}, \ref{fig:p50} y \ref{fig:p100} muestran corridas de una epidemia para distintos valores de $p$, mientras que la figura \ref{fig:maxinf_proba} muestra el impacto que tiene variar esta probabilidad en el porcentaje máximo de infectados.
	
	\begin{figure}
		\centering
		\includegraphics[width=1\linewidth]{Images/25}
		\caption{Ejemplo de una corrida, con probabilidad de contagio al contacto de $p = 0.25$, sin cubrebocas.}
		\label{fig:p25}
	\end{figure}

	\begin{figure}
		\includegraphics[width=1\linewidth]{Images/50}
		\caption{Ejemplo de una corrida, con probabilidad de contagio al contacto de $p = 0.5$, sin cubrebocas.}
		\label{fig:p50}
	\end{figure}

	\begin{figure}
		\centering
		\includegraphics[width=1\linewidth]{Images/75}
		\caption{Ejemplo de una corrida, con probabilidad de contagio al contacto de $p = 0.75$, sin cubrebocas.}
		\label{fig:p75}
	\end{figure}

	\begin{figure}
		\includegraphics[width=1\linewidth]{Images/100}
		\caption{Ejemplo de una corrida, con probabilidad de contagio al contacto de $p = 1$, sin cubrebocas.}
		\label{fig:p100}
	\end{figure}

	\begin{figure}
		\includegraphics[width=1\linewidth]{Images/boxplot6}
		\caption{Gráfico de cajas variando la probabilidad de contagio.}
		\label{fig:maxinf_proba}
	\end{figure}

	\section{Conclusiones}
	\begin{table}
		\centering
		\label{pvalue}
		\caption{Valores $p$ obtenidos al realizar una prueba de T de Student.}
		\begin{tabular}{|c|r|}
			\hline 
			Conjuntos & Valor $p$ \\ 
			\hline 
			1 - 2 & 0.2065 \\ 
			\hline 
			2-3 & 0.0468 \\ 
			\hline 
			3-4 & 0.3030 \\ 
			\hline 
			4-5 & 0.6909 \\ 
			\hline 
		\end{tabular} 
	\end{table}

	En el presente se trabajó con una simulación utilizando un sistema multiagente para verificar si existe un cambio significativo en el porcentaje de agentes contagiados si estos hacen uso de cubrebocas, y si el virus es más o menos contagioso. La figura \ref{conmasc2} muestra que sí existen mejoras en el uso de los cubrebocas, para comprobarlo se realizan pruebas T de Student para revisar si hay diferencia significativa entre las medias de los conjuntos de datos obtenidos. El cuadro \ref{pvalue} muestra los valores $p$ obtenidos de dichas pruebas. Se concluye que cuando la probabilidad de que no traigan cubrebocas es muy grande, se tiene una diferencia significativa en el porcentaje máximo de contagiados.
	
	\subsection{Trabajo futuro}
	
	A este modelo le hace falta tomar en cuenta muchos factores, como lo son el tipo de enfermedad (epidemia) con la que se trata, algunas características de los agentes como el hecho de si poseen alguna enfermedad (cáncer, asma, diabetes, etcétera) que haga que el contagio sea más grave para él. Se podría también cambiar la estructura de la población a algo más realista, como una red \citep{Venkatramanan_Lewis_Chen_Higdon_Vullikanti_Marathe_2018}. Podría también ampliarse el entorno de los agentes a no una sola \textit{región}, es decir, que existan los \textit{viajes} entre diferentes zonas donde cada una tenga una diferente probabilidad de contagio. Referente al tema del uso del cubrebocas, también se podría considerar si el agente hace uso correcto de este o qué tipo utiliza. También puede incorporarse el aislamiento social al modelo. 
	
	\bibliographystyle{elsarticle-num-names}
	\bibliography{Referencias}
	
	
	
	
\end{document}