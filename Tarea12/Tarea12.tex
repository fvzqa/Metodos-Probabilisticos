\documentclass[12pt,letterpaper]{article}
\usepackage[utf8]{inputenc}
\usepackage{tikz}
\usetikzlibrary{trees}
\usepackage[spanish, es-nodecimaldot]{babel}
\usepackage{amsmath}
\usepackage{color}
\usepackage{algorithm}
\usepackage[noend]{algpseudocode}
\renewcommand{\algorithmicrequire}{\textbf{Entrada:}}
\renewcommand{\algorithmicensure}{\textbf{Salida:}}
\usepackage{subcaption}
\usepackage{amsfonts}
\usepackage{hyperref}
 \hypersetup{
     colorlinks=true,
     linkcolor=blue,
     filecolor=blue,
     citecolor = blue,      
     urlcolor=cyan,
     }
\usepackage{amssymb}
\usepackage{listings}
\usepackage{color}

\newcommand\var[1]{\, \mathrm{Var}\lbrack #1 \rbrack}

\newcommand\cov[1]{\, \mathrm{Cov} \lbrack #1 \rbrack}

\newcommand\esp[1]{\, \mathbb{E} \lbrack #1 \rbrack}
\newcommand\integral[4]{\ensuremath{\int_{#1}^{#2} #3 \, d#4}}

\definecolor{mygreen}{rgb}{0,0.6,0}
\definecolor{mygray}{rgb}{0.5,0.5,0.5}
\definecolor{mymauve}{rgb}{0.58,0,0.82}

\lstset{ 
  backgroundcolor=\color{white},   % choose the background color; you must add \usepackage{color} or \usepackage{xcolor}; should come as last argument
  basicstyle=\footnotesize,        % the size of the fonts that are used for the code
  breakatwhitespace=false,         % sets if automatic breaks should only happen at whitespace
  breaklines=true,                 % sets automatic line breaking
  captionpos=b,                    % sets the caption-position to bottom
  commentstyle=\color{mygreen},    % comment style
  deletekeywords={...},            % if you want to delete keywords from the given language
  escapeinside={\%*}{*)},          % if you want to add LaTeX within your code
  extendedchars=true,              % lets you use non-ASCII characters; for 8-bits encodings only, does not work with UTF-8
  firstnumber=1,                % start line enumeration with line 1000
  frame=single,	                   % adds a frame around the code
  keepspaces=true,                 % keeps spaces in text, useful for keeping indentation of code (possibly needs columns=flexible)
  keywordstyle=\color{blue},       % keyword style
  language=R,                 % the language of the code
  morekeywords={*,...},            % if you want to add more keywords to the set
  numbers=none,                    % where to put the line-numbers; possible values are (none, left, right)
  numbersep=5pt,                   % how far the line-numbers are from the code
  numberstyle=\tiny\color{mygray}, % the style that is used for the line-numbers
  rulecolor=\color{black},         % if not set, the frame-color may be changed on line-breaks within not-black text (e.g. comments (green here))
  showspaces=false,                % show spaces everywhere adding particular underscores; it overrides 'showstringspaces'
  showstringspaces=false,          % underline spaces within strings only
  showtabs=false,                  % show tabs within strings adding particular underscores
  stepnumber=2,                    % the step between two line-numbers. If it's 1, each line will be numbered
  stringstyle=\color{mymauve},     % string literal style
  tabsize=2,	                   % sets default tabsize to 2 spaces
  title=\lstname                   % show the filename of files included with \lstinputlisting; also try caption instead of title
}

\usepackage{amsthm}
\newtheorem{theorem}{Teorema}

\usepackage{graphicx}
\usepackage[inner=1.5 cm, outer = 1.5 cm, top=1 cm, bottom = 1.5 cm]{geometry}
\setlength{\parskip}{3mm}
\title{\textsc{Funciones generadoras \\ Ejercicios}}
\author{\textsc{Fabiola Vázquez}}

\setlength{\parindent}{0cm}
\renewcommand{\lstlistingname}{Código}
\floatname{algorithm}{Algoritmo}
\newtheorem{ej}{Ejercicio}
\newtheorem{defi}{Definición}
\newtheorem{teo}{Teorema}



\begin{document}
\maketitle
\hrule 
\begin{ej}[Ej. 1, pág. 392]
$Z_1, Z_2, \ldots, Z_n$ describen un proceso en el que cada padre tiene descendencia con probabilidad $p_j$. Encuentra la probabilidad $d$ de que el proceso termine si
\begin{itemize}
\item[(a)] $p_0 = \frac{1}{2}, p_1 = \frac{1}{4}, p_2 = \frac{1}{4}.$
\item[(b)] $p_0 = \frac{1}{3}, p_1 = \frac{1}{3}, p_2 = \frac{1}{3}.$
\item[(c)] $p_0 = \frac{1}{3}, p_1 = 0, p_2 = \frac{2}{3}.$
\item[(d)] $p_j = \frac{1}{2^{j+1}}$, para $j = 0, 1, 2, \ldots$.
\item[(e)] $p_j = \left(\frac{1}{3}\right)\left(\frac{2}{3}\right)^j$, para $j = 0, 1, 2, \ldots$.
\item[(f)] $p_j = e^{-2}\left(\frac{2^j}{j!}\right)$, para $j = 0, 1, 2, \ldots$.
\end{itemize}
\end{ej}
\begin{proof}[Solución]

\textbf{(a)} Se calcula la media de la descendencia, 
\begin{equation}
m= \frac{1}{4} + 2\left(\frac{1}{4}\right) = \frac{3}{4}.
\end{equation}
Como $m<1$ por el teorema 10.2 \cite{snell}, el proceso termina con probabilidad $d=1$.

\textbf{(b)} Se calcula la media de la descendencia, 
\begin{equation}
m= \frac{1}{3} + 2\left(\frac{1}{3}\right) = \frac{3}{3} = 1.
\end{equation}
Como $m=1$ por el teorema 10.2 \cite{snell}, el proceso termina con probabilidad $d=1$.

\textbf{(c)} Se calcula la media de la descendencia, 
\begin{equation}
m= 0 + 2\left(\frac{2}{3}\right) = \frac{4}{3}.
\end{equation}
Como $m>1$, se considera la ecuación $h(z) = z$ y se encuentran las raíces,
\begin{align}
h(z) &= z \\
p_0 + p_1 z + p_2 z^2 + \ldots &= z \\
\frac{1}{3} + 0\cdot z + \frac{2}{3} \cdot z^2 &= z \\
\frac{1}{3} + \frac{2}{3} \cdot z^2 - z &= 0\\
1 + 2z^2 -3z &=0 \\
(2z -1)(z - 1) &=0
\end{align}
Entonces, las raíces de la ecuación son $z=\frac{1}{2}$ y $z=1$, donde $d$ es la menor de las raíces de $h(z)=z$, por lo tanto $d=\frac{1}{2}$.

\textbf{(d)} Se calcula la media de la descendencia, 
\begin{align}
m &= \sum_{j=0}^{\infty} \frac{j}{2^{j+1}} \\
&= \sum_{j=1}^{\infty} \frac{(j-1)}{2^j} =  \sum_{j=1}^{\infty} \frac{j}{2^j} - \sum_{j=1}^{\infty} \frac{1}{2^j} \\
&= \frac{\frac{1}{2}}{\left(1 - \frac{1}{2}\right)^2} - \frac{\frac{1}{2}}{1-\frac{1}{2}} \\
&= \frac{\frac{1}{2}}{\left(\frac{1}{2}\right)^2} - \frac{\frac{1}{2}}{\frac{1}{2}} = \frac{1}{\frac{1}{2}} - 1 = 2 - 1 = 1.
\end{align}
Dado que $m=1$, se concluye que $d=1.$

\textbf{(e)} Se calcula el valor de $m$,
\begin{align}
m &= \sum_{j=0}^{\infty} \frac{j}{3}  \left(\frac{2}{3}\right)^j \\
&= \frac{1}{3} \cdot \frac{\frac{2}{3}}{\left(1-\frac{2}{3}\right)^2} = \frac{1}{3} \cdot \frac{\frac{2}{3}}{\frac{1}{9}} = \frac{1}{3} \cdot 6 = 2.
\end{align}
Como $m > 1$, se considera la ecuación $h(z)=z$ y se buscan sus raíces,
\begin{align}
h(z) &= z \\
p_0 + p_1 \cdot z + p_2 \cdot z^2 + \ldots &= z \\
\sum_{j=0}^{\infty} \left(\frac{1}{3}\right) \left(\frac{2}{3}\right)^j \cdot z^j &=z.
\end{align}
Como $|z|<1$, entonces $|\frac{2}{3}z| < 1$ y 
\begin{align}
\left(\frac{1}{3}\right) \sum_{j=0}^{\infty} \left(\frac{2}{3}\right)^j \cdot z^j &=z \\
\left(\frac{1}{3}\right) \cdot \frac{1}{1 - \frac{2}{3}z} &=z \\
\frac{1}{3} \cdot \frac{1}{\frac{3-2z}{3}} &=z \\
\frac{1}{3-2z} &=z \\
2z^2 -3z +1 &=0 \\
(2z-1)(z-1)&=0,
\end{align}
por lo que las raíces de la ecuación $h(z)=z$ son $z_1=0$ y $z_2=\frac{1}{2}$, por lo tanto $d=\frac{1}{2}$.

\textbf{(f)} Se calcula el valor de $m$,
\begin{align}
m &= \sum_{j=0}^{\infty} \frac{e^{-2}2^j}{j!}\cdot j \\
&= e^{-2} \sum_{j=0}^{\infty} \frac{2^j}{j!} \cdot j.
\end{align}
Dado que el primer término de la suma es igual a cero, 
\begin{align}
m &= e^{-2} \sum_{j=1}^{\infty} \frac{2^j}{j!} \cdot j \\
&= e^{-2} \sum_{j=1}^{\infty} \frac{2^j}{(j-1)!} = e^{-2} \sum^{\infty}_{j=1} \frac{2\cdot2^{j-1}}{(j-1)!} \\
&= 2e^{-2} \sum^{\infty}_{j=1} \frac{2^{j-1}}{(j-1)!} \\
&= 2e^{-2} \sum^{\infty}_{j=0} \frac{2^{j}}{j!} \\
&= 2e^{-2}e^{2} = 2.
\end{align}
Como $m>1$, se considera $h(z) = z$ y se buscan sus raíces,
\begin{align}
h(z) &= z \\
p_0 + p_1 \cdot z + p_2 \cdot z^2 + \ldots &= z \\
\sum_{j=0}^{\infty} \frac{e^{-2}2^j}{j!}\cdot z^j &= z \\
e^{-2} \sum_{j=0}^{\infty} \frac{(2z)^j}{j!} &= z \\
e^{-2}\cdot e^{2z} &= z\\
e^{2z-2} &=z.
\end{align}
Calculando las raíces de la ecuación por medio de Wolfram Alpha \cite{Wolfram|Alpha}, tenemos que $z_1=1$ y $z_2 \approx 0.2$, por lo tanto $d \approx 0.2$.
\end{proof}

\begin{ej}[Ej. 3, pág. 392]
En el problema de la carta en cadena, encuentra la ganancia esperada si
\begin{enumerate}
\item[(a)] $p_0 = \frac{1}{2}, p_1 = 0$ y $p_2 = \frac{1}{2}$.
\item[(b)] $p_0 = \frac{1}{6}, p_1 = \frac{1}{2}$ y $p_2 = \frac{1}{3}$.
\end{enumerate}
Demuestra que si $p_0 > \frac{1}{2},$ no puedes esperar ganancias.
\end{ej}
\begin{proof}[Solución]
En el ejemplo 10.14, se menciona que la ganancia esperada viene dada por,
\begin{equation}
\label{ganancia}
50m + 50m^{12} - 100,
\end{equation} 
donde $m = 0 \cdot p_0 +1\cdot p_1 + 2 \cdot p_2$. Para que la  ganancia sea favorable, se debe tener que 
\begin{equation}
m+m^{12} > 2.
\end{equation}
Lo cual es cierto si y solo si $m>1$, es decir, si
\begin{align}
p_1 + 2p_2 > 1.
\end{align}
Recordando que $p_0 + p_1 + p_2 = 1$, entonces $p_1 = 1-p_0 - p_2$, sustituyendo resulta,
\begin{align}
1-p_0 -p_2 + 2p_2 &>1 \\
p_2 - p_0 &>0 \\
p_2 &>p_0.
\end{align}
Si $p_0 = \frac{1}{2}$, el valor más grande que puede tomar $p_2$ es $\frac{1}{2}$ (si se hace $p_1 =0$), pero no cumpliría que $p_2 > p_0$, por lo que no sería un valor favorable. Por lo tanto si $p_0 = \frac{1}{2}$, no se puede esperar una ganancia.

\textbf{(a)} Se calcula el valor de $m$ como
\begin{align}
m = 0\cdot \frac{1}{2} + 1\cdot 0 + 2\cdot\frac{1}{2}= 1, 
\end{align}
por la ecuación \ref{ganancia}, se tiene que la ganancia esperada es $50(1)+50(1^{12})-100 = 0$.

\textbf{(b)} De forma similar que en el inciso anterior, se tiene el valor de $m$,
\begin{equation}
m = 0 \cdot \frac{1}{6} + 1 \cdot \frac{1}{2} + 2\cdot \frac{1}{3} = \frac{7}{6},
\end{equation}
por lo tanto la ganancia esperada es $50\left(\frac{7}{6}\right)+50\left(\left(\frac{7}{6}\right)^{12}\right)-100 \approx 276.26$.
\end{proof}

\begin{ej}[Ej. 1, pág. 401]
Sea $X$ una variable aleatoria continua con valores en $[0,2]$ y densidad $f_{X}$. Encuentra la función generadora de momentos $g(t)$ de $X$ si
\begin{enumerate}
\item[(a)] $f_{X} = \frac{1}{2}$.
\item[(b)] $f_{X} = \frac{1}{2}x$.
\item[(c)] $f_{X} = 1 - \frac{1}{2}x$.
\item[(d)] $f_{X} = |1-x|$.
\item[(e)] $f_{X} = \frac{3}{8}x^{2}.$
\end{enumerate}
\end{ej}

\begin{proof}[Solución]
La función generadora de momentos $g(t)$ de $X$  se define \cite{snell} como
\begin{equation}
g(t) = \int_{-\infty}^{\infty}e^{tx} f_{X}(x) dx.
\end{equation}

\textbf{(a)} Calculamos $g(t)$,
\begin{align}
g(t) &= \int_{0}^{2} e^{tx} \cdot \frac{1}{2} dx \\
&= \frac{1}{2} \int_{0}^{2} e^{tx} dx \\
&= \frac{1}{2} \left( \frac{e^{tx}}{t} \, \,\Big\lvert^{2}_{0}\right) = \frac{1}{2} \left( \frac{e^{2t}}{t} - \frac{e^{0t}}{t} \right) = \frac{1}{2} \left( \frac{e^{2t}}{t} - \frac{1}{t}\right) \\ 
&= \frac{1}{2} \cdot \frac{e^{2t}-1}{t} = \frac{e^{2t}-1}{2t}.
\end{align}
Por lo tanto si $f_X = \frac{1}{2}$, su función generadora de momentos es $g(t)=\frac{e^{2t}-1}{2t}$.

\textbf{(b)} Se calcula $g(t)$,
\begin{align}
g(t) &= \int_{0}^{2} e^{tx} \cdot \frac{1}{2} x dx = \frac{1}{2}\int_{0}^{2} xe^{tx}dx, 
\end{align}
se hace una integración por partes, haciendo $u=x$ y $dv=e^{tx}dx$, entonces,
\begin{align}
g(t) &= \frac{1}{2} \left( \frac{xe^{tx}}{t}\,\,\Big\lvert ^{2}_{0} - \int^{2}_{0} \frac{e^{tx}}{t}dx\right) \\
&= \frac{1}{2} \left( \left(\frac{xe^{tx}}{t} - \frac{e^{tx}}{t^{2}} \right)\Big\lvert ^{2}_{0} \right) \\
&= \frac{1}{2} \left( \frac{2e^{2t}}{t} - \frac{e^{2t}}{t^{2}} - \frac{0e^{0t}}{t} + \frac{e^{0t}}{t^2}\right) \\ 
&= \frac{e^{2t}}{t} + \frac{1}{2t^2} - \frac{e^{2t}}{2t^2}.
\end{align}
Por lo tanto, si $f_{X} = \frac{1}{2}x$, su función generadora de momentos es $g(t) = \frac{e^{2t}}{t} + \frac{1}{2t^2} - \frac{e^{2t}}{2t^2}.$

\textbf{(c)} Se calcula la función $g(t)$,
\begin{align}
g(t) &= \int^{2}_{0} e^{tx} \left(1-\frac{1}{2}x\right)dx = \underbrace{\int^{2}_{0} e^{tx}dx}_{(1)} - \underbrace{\frac{1}{2} \int^{2}_{0} xe^{tx}dx}_{(2)}.
\end{align}
En el inciso \textbf{(a)} se calculó el valor de la integral (1) y en el inciso \textbf{(b)} se calculó la integral (2), por tanto
\begin{align}
g(t) &= \frac{e^{tx}}{t} \Big\lvert^{2}_{0} - \left(\frac{e^{2t}}{t} + \frac{1}{2t^2} - \frac{e^{2t}}{2t^2}\right) \\
& = \frac{e^{2t}}{t} - \frac{1}{t} - \frac{e^{2t}}{t} - \frac{1}{2t^2} + \frac{e^{2t}}{2t^2} \\
&= \frac{e^{2t}}{t^{2}} - \frac{1}{2t^2} - \frac{1}{t}.
\end{align}
Por lo tanto, si $f_{X}=1-\frac{1}{2}x$, entonces su función generadora de momentos es $g(t) = \frac{e^{2t}}{t^{2}} - \frac{1}{2t^2} - \frac{1}{t}.$

\textbf{(d)} Se calcula la función generadora de momentos $g(t)$,
\begin{align}
g(t) &= \int^{2}_{0} e^{tx} |1-x| dx,
\end{align}
donde la función $|1-x|$, se define como 
\begin{equation}
|1-x|= \left\{ \begin{array}{lcc}
             1-x &   \text{si}  & 1 \geqslant x, \\
             \\ x-1 &  \text{si} & x > 1. 
             \end{array}
   \right.
\end{equation}
Entonces,
\begin{align}
g(t) &= \int^{1}_{0} e^{tx}(1-x)dx + \int^{2}_{1} e^{tx}(x-1)dx \\
&= \int^{1}_0 e^{tx}dx - \int^{1}_{0} xe^{tx}dx + \int^{2}_{1}xe^{tx}dx - \int^{2}_{1}e^{tx}dx \\
&= \frac{e^{tx}}{t} \Big\lvert^{1}_{0} - \left(\frac{xe^{tx}}{t} - \frac{e^{tx}}{t^2} \Big\lvert^{1}_{0} \right) + \left(\frac{xe^{tx}}{t} - \frac{e^{tx}}{t^2} \Big\lvert^{2}_{1} \right) - \frac{e^{tx}}{t} \Big\lvert^2_0 \\
&= \frac{2e^{t}}{t^2} + \frac{e^{2t}}{t} - \frac{e^{2t}}{t^2} - \frac{1}{t^2} - \frac{1}{t}.
\end{align}

\textbf{(e)} Se calcula la función generadora de momentos $g(t)$,
\begin{align}
g(t) &= \integral{0}{2}{e^{tx} \frac{3}{8} x^2}{x} \\ 
&= \frac{3}{8} \underbrace{\integral{0}{2}{x^2 e^{tx}}{x}}_{(3)}.
\end{align}
La integral (3) se resuelve por partes, haciendo $u=x^2$ y $dv=e^{tx}dx$, queda
\begin{align}
g(t) &= \frac{x^2e^{tx}}{t} \Big\lvert^{2}_{0} - \underbrace{\frac{2}{t} \integral{0}{2}{xe^{tx}}{x}}_{(4)}.
\end{align}
Nuevamente, la integral (4) se resuelve por partes, haciendo $u=x$ y $dv=e^{tx}dx$, resulta
\begin{align}
g(t) &= \frac{x^2e^{tx}}{t} \Big\lvert^{2}_{0} - \frac{2}{t}\left(\frac{xe^{tx}}{t} \Big\lvert^{2}_{0}- \frac{1}{t} \integral{0}{2}{e^{tx}}{x}\right) \\
&= \frac{x^2e^{tx}}{t} \Big\lvert^{2}_{0} - \frac{2}{t}\left(\frac{xe^{tx}}{t} \Big\lvert^{2}_{0}- \frac{e^{tx}}{t^2} \Big\lvert^{2}_{0}  \right) \\
&= \frac{3e^{2t}}{2t} - \frac{3e^{2t}}{2t^2} + \frac{3e^{2t}}{4t^3} - \frac{3}{4t^3}.
\end{align}
\end{proof}

\begin{ej}[Ej. 6, pág. 402]
Sea $X$ una variable aleatoria continua con función característica $k_{X}(\tau)=e^{-|\tau|}, -\infty < \tau < \infty$. Demuestra que la función de densidad de $X$ es $f_{X}(x)=\frac{1}{\pi(1-x^2)}$.
\end{ej}
\begin{proof}[Solución]
Dada la función $k_{X}(\tau)$ se puede obtener la función de densidad de $X$ \cite{snell} con
\begin{equation}
f_{X}(x) = \frac{1}{2\pi}\integral{-\infty}{\infty}{e^{-i\tau x}k_{X}(\tau)}{\tau}.
\end{equation}
Entonces, sustituyendo la función $k_{X}(\tau)$, se tiene
\begin{equation}
\label{int}
f_{X}(x) = \frac{1}{2\pi} \integral{-\infty}{\infty}{e^{-i\tau x} \cdot e^{-|\tau|}}{\tau}.
\end{equation}
Se sabe que $e^{ix}=\cos x + i \sen x$, entonces $e^{i(-\tau x)}=\cos(-\tau x) + i \sen (-\tau x)$ y como la función $\cos(x)$ es una función par, entonces $\cos (-x) = \cos (x)$. La función $\sen(x)$ es una función impar, entonces $\sen(-x)=-\sen(x)$. Entonces $e^{i(-\tau x)} = \cos(\tau x) - i\sen(\tau x)$, sustituyendo en la ecuación \ref{int} se tiene que
\begin{align}
f_{X}(x) &=\frac{1}{2\pi} \integral{-\infty}{\infty}{ (\cos (\tau x) - i\sen (\tau x))e^{-|	\tau|}}{\tau} \\
&=\frac{1}{2	\pi}\left[ \integral{-\infty}{\infty}{ \cos (\tau x) e^{-|\tau|}}{\tau} - i \integral{-\infty}{\infty}{ \sen (\tau x) e^{-|\tau|}}{\tau}\right].
\end{align}
La función $e^{-|\tau|}$, se define como
\begin{equation}
e^{-|\tau|}= \left\{ \begin{array}{lcc}
             e^{\tau} &   \text{si}  & \tau \geqslant 0, \\
             \\ e^{-\tau} &  \text{si} & \tau < 0. 
             \end{array}
   \right.
\end{equation}
Entonces,
\begin{align}
f_{X}(x) &= \frac{1}{2\pi}\left[\integral{-\infty}{0}{e^{-\tau}\cos(\tau x)}{\tau} + \integral{0}{\infty}{e^{\tau}\cos(\tau x)}{\tau} - i\integral{-\infty}{0}{e^{-\tau}\sen(\tau x)}{\tau} -i \integral{0}{\infty}{e^{\tau}\sen(\tau x)}{\tau}\right].
\end{align}
Como la función coseno es una función par, se tiene
\begin{align}
\integral{-\infty}{0}{e^{-\tau}\cos(\tau x)}{\tau} & = \integral{0}{\infty}{e^{-(-\tau)}\cos(-\tau x)}{\tau} \\
&= \integral{0}{\infty}{e^{\tau}\cos(\tau x)}{\tau}.
\end{align}

Ya que la función seno es una función impar, se tiene
\begin{align}
\integral{-\infty}{0}{e^{-\tau}\sen(\tau x)}{\tau} & = \integral{0}{\infty}{e^{-(-\tau)}\sen(-\tau x)}{\tau} \\
&= -\integral{0}{\infty}{e^{\tau}\sen(\tau x)}{\tau}.
\end{align}

Sustituyendo, 
\begin{align}
f_{X}(x) &=\frac{1}{2\pi} \left(2 \integral{0}{\infty}{e^{\tau}\cos(\tau x)}{\tau}\right) \\
&= \frac{1}{\pi}  \integral{0}{\infty}{e^{\tau}\cos(\tau x)}{\tau}. \label{int2}
\end{align}
La integral se resuelve por partes haciendo $u=\cos \tau x$ y $dv = e^{\tau}d\tau$, entonces $du=-x\sin \tau x$ y $v=e^\tau$, por lo que 
\begin{align}
\integral{}{}{e^{\tau}\cos(\tau x)}{\tau} = e^\tau\cos \tau x + x \int e^\tau \sen \tau x \,\,d\tau. 
\end{align}
Nuevamente, se realiza el proceso de integración por partes haciendo $u=\sen \tau x$ y $dv = e^\tau d\tau$, entonces
\begin{align}
\integral{}{}{e^{\tau}\cos(\tau x)}{\tau} &= e^\tau\cos \tau x + x\left( \sen \tau x \cdot e^\tau - x \int e^\tau \cos \tau x d\tau \right) \\ &= e^\tau\cos \tau x + x \sen \tau x \cdot e^\tau - x^2 \int e^\tau \cos \tau x d\tau \\
\integral{}{}{e^{\tau}\cos(\tau x)}{\tau} + x^2 \integral{}{}{e^{\tau}\cos(\tau x)}{\tau} &= e^\tau\cos \tau x + x \sen \tau x \cdot e^\tau \\
(1+x^2)\integral{}{}{e^{\tau}\cos(\tau x)}{\tau} &= e^\tau\cos \tau x + x \sen \tau x \cdot e^\tau \\
\integral{}{}{e^{\tau}\cos(\tau x)}{\tau} &=\frac{e^\tau\cos \tau x + x \sen \tau x \cdot e^\tau}{1+x^2} \\
\end{align}
Finalmente, sustituyendo en la ecuación \ref{int2}, se tiene
\begin{align}
f_{X}(x) &= \frac{1}{\pi}  \left( \frac{e^\tau\cos \tau x + x \sen \tau x \cdot e^\tau}{1+x^2} \,\,\, \Big\lvert_{0}^{\infty} \right) \\ &=\frac{1}{\pi} \left( 0 - \frac{1}{1+x^2} \right) = \frac{1}{\pi(1+x^2)}
\end{align}
\end{proof}

\begin{ej}[Ej. 10, pág. 402] 
Sea $X_1, X_2, \ldots, X_n$ un proceso de ensayos independientes con densidad 
\begin{equation}
f(x)= \frac{1}{2}e^{-|x|}, \qquad \quad -\infty < x < \infty.
\end{equation}
\begin{enumerate}
\item[(a)] Encuentra la media y la varianza de $f(x)$.
\item[(b)] Encuentra la función generadora de momentos para $X_1, S_n, A_n$ y $S^*_n$.
\item[(c)] ¿Qué puedes decir acerca de la función generadora de momentos de $S^*_n$ cuando $n\rightarrow \infty$?
\item[(d)] ¿Qué puedes decir acerca de la función generadora de momentos de $A_n$ cuando $n\rightarrow \infty$?
\end{enumerate}
\end{ej}

\begin{proof}[Solución]
\textbf{(a)} Para encontrar la media y la varianza de $f(x)$, primero se calcula la función generadora de momentos de la variable aleatoria $X_1$. Por el teorema 10.4 \cite{snell}, se tiene la relación
	\begin{equation}
		k_X (\tau) = g_X (i \tau) = \int_{-\infty}^{\infty} e^{i \tau x} f_X (x) dx =\frac{1}{2} \int_{-\infty}^{\infty} e^{i \tau x} e^{-|x|} \, dx.
	\end{equation}
En el ejercicio 4, se resolvió una integral similar, cuyo valor encontrado es
\begin{equation}
\frac{1}{2\pi} \integral{-\infty}{\infty}{e^{-i\tau x} \cdot e^{-|\tau|}}{\tau} = \frac{1}{\pi(1+x^2)}
\end{equation}
Se le da la forma para obtener la integral deseada. Primero, se multiplica ambos lados por $2\pi$,
\begin{align}
\integral{-\infty}{\infty}{e^{-i\tau x} \cdot e^{-|\tau|}}{\tau} = \frac{2}{(1+x^2)},
\end{align}
después se cambia $x$ por $xi$, entonces
\begin{align}
 \integral{-\infty}{\infty}{e^{-i\tau (xi)} \cdot e^{-|\tau|}}{\tau} &= \frac{2}{(1+(xi)^2)} \\
\integral{-\infty}{\infty}{e^{\tau x} \cdot e^{-|\tau|}}{\tau} &= \frac{2}{1-x^2} .
\end{align}

Cambiando $x$ por $t$ y $\tau$ por $x$, se tiene que  
\begin{align}
g_X(t) &= \frac{1}{2} \integral{-\infty}{\infty} {e^{tx} e^{-|x|}}{x}\\
&= \frac{1}{2}\cdot \frac{2}{1-t^2} \\ &=\frac{1}{1-t^2}. 
\end{align}
Por lo tanto, la función generadora de momentos de $X_1$ es la función $g(t)=\frac{1}{1-t^2}$.

La media se obtiene derivando la función generadora de momentos y evaluándola en $t=0$,
	\begin{align}
		g'_X(t) &= \frac{2t}{(1-t^2)^2} \\
		g'_X(0) &= \frac{2\cdot 0}{(1-0^2)^2} = 0,
	\end{align}
para la varianza se hace lo mismo, pero en la segunda derivada,
	\begin{align}
	g''_X(t) &= \frac{2}{(1-t^2)^2} + \frac{8t^2}{(1-t^2)^3} \\
	g''_X(0) &= \frac{2}{(1-0^2)^2} + \frac{8\cdot 0^2}{(1-0^2)^3} = 2. 
	\end{align}

Por lo tanto, la media es igual a 0 y la varianza es igual a 2.

\textbf{(b)} La función generadora de momentos de $X_1$ se obtuvo en el inciso anterior, y es $g_X (t) = \frac{1}{1 - t^2}$. Entonces, se calcula la función generadora de momentos para $S_n,$ 
	\begin{align}
		g_{S_n} (t) = \esp{e^{S_n t}} &= \esp{e^{(X_1 + \dots + X_n) t}}\\
				&= \esp{e^{X_1 t+ X_2 t + \cdots X_n t}}\\
		&= \esp{e^{X_1 t} e^{X_2 t} \cdots e^{X_n t}} \\
		&= \esp{e^{X_1 t}} \cdots \esp{e^{X_n t}} \\
		&= g_{X_1} (t) \cdots g_{X_n} (t) \\
		&= \left(\frac{1}{1-t^2}\right)^{n}.
	\end{align}
De forma similar, se obtiene la función generadora de momentos para $A_n = \frac{S_n}{n}$,  
	\begin{align}
		g_{A_n} (t) &= g_{ \frac{S_n}{n}}(t) \\
		& = \esp{ e^{\frac{S_n}{n} t}} \\ &=  \esp{ e^{S_n \frac{t}{n}}}\\ &= g_{S_n} \left(\frac{t}{n}\right)\\
		&= \left( \frac{1}{1-\left(\frac{t}{n}\right)^2} \right)^n.
	\end{align}
	Para encontrar la función generadora de momentos de $S_{n}^{*} = \frac{S_n - n\mu}{\sqrt{n \sigma^2}}$ se realiza un proceso similar que el anterior. Primero se sustituye la media y la varianza encontrada en el inciso \textbf{(a)}, por lo que se tiene 
	\begin{equation}
	S_{n}^{*}  = \frac{S_n}{\sqrt{2 n}}.
	\end{equation}
Entonces,
	\begin{equation}
		g_{S_{n}^{*}} (t) = g_{\frac{S_n}{\sqrt{2n}}} (t)=  g_{S_n} \left(\frac{t}{\sqrt{2n}}\right) =  \left( \frac{1}{1-\left(\frac{t}{\sqrt{2n}}\right)^2} \right)^n.
	\end{equation}

\textbf{(c)} Se calcula el límite de $g_{S^*_n}$ cuando $n\longrightarrow \infty$,
\begin{equation}
	\lim\limits_{n \longrightarrow \infty} g_{S^*_n} = \lim_{n \longrightarrow \infty} \left( \frac{1}{1-\left(\frac{t}{\sqrt{2n}}\right)^2} \right)^n = \frac{\lim\limits_{ n \longrightarrow \infty} 1^n}{\lim\limits_{n \longrightarrow \infty} \left(1-\left(\frac{t}{\sqrt{2n}}\right)^2\right)^n}.
\end{equation}
Se sabe que si una sucesión $(a_n)$ converge a $a$, entonces $(1 + \frac{a_n}{n})^n$ converge a $e^a$ \cite{Casella}. Entonces,
\begin{equation}
	\lim\limits_{n \longrightarrow \infty} \left(1-\left(\frac{t}{\sqrt{2n}}\right)^2\right)^n = \lim\limits_{n \longrightarrow \infty} \left(1+\left(\frac{\frac{-t^2}{2}}{n}\right)\right)^n =e^{\frac{-t^2}{2}}.
\end{equation}
Sustituyendo, el límite resulta
\begin{equation}
	\lim\limits_{n\longrightarrow \infty} g_{S^*_n} = \lim\limits_{n \longrightarrow \infty} \left( \frac{1}{1-\left(\frac{t}{\sqrt{2n}}\right)^2} \right)^n = \frac{\lim\limits_{n \longrightarrow \infty} 1^n}{\lim\limits_{n \longrightarrow \infty} \left(1-\left(\frac{t}{\sqrt{2n}}\right)^2\right)^n} = \frac{1}{e^{\frac{-t^2}{2}}} = e^{\frac{t^2}{2}}.
\end{equation}
\textbf{(d)} De forma similiar, se calcula
	\begin{equation}
		\lim\limits_{n \longrightarrow \infty} g_{A_n} =\lim_{n \longrightarrow \infty}  \left( \frac{1}{1-\left(\frac{t}{n}\right)^2} \right)^n = \frac{\lim\limits_{n \longrightarrow \infty} 1^n}{\lim\limits_{n \longrightarrow \infty} \left( 1-\left(\frac{t}{n}\right)^2 \right)^n},
	\end{equation}
	donde, 
\begin{equation}
	1-\left(\frac{t}{n}\right)^2 = 1 + \left( \frac{-t^2}{n^2}\right) = 1 + \frac{\frac{-t^2}{n}}{n}.
\end{equation}
La sucesión $a_n = \frac{-t^2}{n}$ converge a 0, así que 
\begin{equation}
	\lim\limits_{n \longrightarrow \infty} \left( 1-\left(\frac{t}{n}\right)^2 \right)^n = e^0 = 1,
\end{equation}
Por lo tanto, se tiene que
\begin{equation}
	\lim\limits_{n \longrightarrow \infty}  \left( \frac{1}{1-\left(\frac{t}{n}\right)^2} \right)^n = 1.
\end{equation}
\end{proof}



\bibliographystyle{plain} 
\bibliography{ref}


\end{document} 